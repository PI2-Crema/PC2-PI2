\section{Solução de Software}
O grupo de software ficou responsável pelo desenvolvimento da interação do usuário com o sistema, ou seja, em criar um ambiente onde o usuário possa configurar e analisar os dados capitados pelo sistema.

\section{Solução}
O principal objetivo da solução de software é disponibilizar um ambiente onde o piscicultor possa analisar o rendimento e andamento do trato de peixe que está sendo cultivado.

Com esse objetivo em mente, foi necessário desenvolver também um sistema gerenciador de estoque. Dessa forma é possivel registrar o investimento feito em cada trato de peixe. Possibilitando assim uma análise de qual ração tem o melhor custo benefício no cultivo de peixe.

Outro fator que foi levado em consideração são os paramentros que são analisados em uma piscicultura, isto é, os valores de PH, temperatura, conductividade e a taxa de conversão alimentar(TCA) em relação ao desenvolvimento e custo do trato de peixe.

\section{Arquitetura Geral}

\begin{figure}[H]
    \centering
    \includegraphics[scale=0.7, angle=90]{arquitetura_geral}
    \caption{Arquitetura Geral}
    \label{fig:arquitetura_geral}
\end{figure}

A arquitetura do sistema foi dividida em três pacotes que pode ser visualizados na figura \ref{fig:arquitetura_geral}. Um módulo que se encontra na placa raspberry, um módulo de gerenciamento de dados(API) e o módulo de interface gráfica com o usuário.

\subsection{Módulo Raspberry}

O módulo que se encontra na raspberry foi desenvolvido em python e a sua função é sincronizar os dados recebidos pela rede \textit{wireless} e a API.

O mecanismo utilizado para a transmisão de dados da rede rede \textit{wireless} e o módulo foi a escrita/leitura de arquivos.

\subsubsection{Módulo de \textit{output}}
Um arquivo de entrada e outro de saida foram definidos, assim como o formato de escrita em cada um deles. O processo de saida de dados da rede \textit{wireless} para a api funciona da seguinte maneira: o móludo de sincronização observa o arquivo de saida de dados do módulo \text{wireless} e quando há mudanças realiza os seguintes procedimentos:

\begin{itemize}
  \item Cria uma cópia do arquivo de saída exluindo o original. Eviando assim uma condição de corrida entre o módulo e a rede \textit{wireless}.
  \item Ler o arquivo cópia, realizando o \textit{parse} para o formato JSON esperado pela API
  \item Ao fim do \textit{parse} realiza o \textit{request} para a API no \textit{endpoint} de registro de dados
  \item Caso o \textit{request} falhe, o arquivo de cópia é mantido para ser utilizado na próxima tentativa de registro. Caso funcione, o arquivo é apagado
\end{itemize}

\subsubsection{Padrão de escrita}

Cada linha do arquivo representa um bloco de dados para cada alimentador na rede e cada dado tem a sua posição na linha separados por espaço. O formato atual do arquivo é apresentado a seguir:\\

ID HORA MINUTO NIVEL-COMIDA NIVEL-BATERIA ERROR-CODE PH TEMPERATURA CONDUTIVIDADE

\subsubsection{Exemplo de arquivo}

1 13 36 53 13 8 14 31 100\\
3 5 10 67 26 6 11 1 30\\
3 2 1 74 60 10 8 -13 21\\
1 5 35 19 17 1 6 36 29\\
1 7 20 37 73 4 10 -11 71\\
3 12 12 44 6 7 7 15 62\\

\subsubsection{Módulo de \textit{input}}

O módulo de envio de dados para os alimentadores através da rede sem fio é feito através da escrita em um arquivo de entrada de dados. Para isso era necessário o desenvolvimento de um mini \textit{http server} capaz de receber um requisição e escrevel-lá no arquivo observado pelo módulo da rede \textit{wireless}.

Sendo assim, foi desenvolvido um mini servidor em python capaz de receber \textit{requests} via \textit{POST} e escrever no arquivo de \textit{input} no formato esperado pelo módulo da rede. O formato do arquivo é apresentado a seguir.

ID HORA MINUTO FREQUENCIA QUANTIDADE

O código do módulo pode ser encontrado \href{https://github.com/PI2-Crema/parse-base-module}{aqui}.

\section{API}

A API é responsável por toda a gerência dos dados do sistema. Tanto dos dados adiquiridos pelos alimentadores, o \textit{status} de cada alimentador como o controle de estoque e tratos de peixes, ou seja, é responsavel por gerir os dados de compra de matéria, lotes de material, os tipos de trato de peixes, os tratos já cultivados, os dados coletados pelos sensores, o \textit{status} de cada alimentador no que diz respeit ao nível de ração e bateria, os tanques de peixe, os registro de desenvolvimento de cada trato.

Para a organização desses dados, a api for desenvolvida baseando-se no seguindo o modelo entidade relacional:

\begin{figure}[H]
    \centering
    \includegraphics[scale=0.7, angle=90]{mer}
    \caption{MER}
    \label{fig:mer}
\end{figure}

Em geral a API fornece uma interface de gerenciamento de entidades, isto é, operações de CRUD.

\subsubsection{Sincronização de Dados}

Ao receber dados do módulo de sincronização, é necessário verificar se esses dados são de alimentadores já existentes no sistema ou não. Designando assim os dados para o alimentador certo, caso já exista ou crie um novo caso seja necessário.

Para essa tipo de comportamento foi estabelecido dois estados para os alimentadores. São eles: 1 - Necessita de Configuração e 2 - Configurado. O estado 1 significa que este alimentador é novo na rede e necessita de configuração adequada enquanto o estado 2 siginifica que este alimentador já existe no sistema.

A API valida o estado de cada dado vindo da rende verificando se os sensores/alimentadores sendo registrados já existem em sua base de dados através de um id de rede.

\subsubsection{Dados gráficos}

Outro tipo de dado fornecido pela API são dados históricos de sensores. Fornecendo em formatação adequada desses dados e com a posibilidade de filtra o periodo desejado.

\subsection{API - Django}

Conforme estabelecido no relatório de projeto 1, a API iria ser escrita em python com a utilização do framework DJANGO. Contudo essa escolha não se concretizou devido ao pouco conhecimento da ferramenta por maioria do time de desenvolvimento. Contudo um código foi desenvolvido para teste e validação em django e pode ser encontrado \href{https://github.com/PI2-Crema/API-django}{aqui}.

\subsection{API - Rails}

Com a queda da escolha da tecnologia seleciona para o desenvolvimento da API, foi selecionado o framework Ruby on Rails. A escolha se deve ao grande conhecimento da maioria da parte da equipe de desenvolvimento. O código da API pode se encontrado \href{https://github.com/PI2-Crema/API-rails}{aqui}.

\section{Frontend}

Conforme especificado no relatório de projeto 1. O frontend foi feito através do framework Emberjs que fornece uma solução robusta apara aplicação \textit{single page}. O código do frontend pode ser encontrado \href{https://github.com/PI2-Crema/front-ember}{aqui}.

\section{Emulação}

Para a emulação de dados, foi desenvolvido no móduda raspbery um função geradora de dados. Assim é possivel emular a escrida de dados pela rede \textit{wireless} na placa e observar o comportamento do sistema.

\section{Teste de integração com hardware}

Foi realizado um teste de inteção com o hardware e a rede \textit{wireless} e o comportamento foi o esperado. Contudo há a necessidade de realizações de mais casos de teste. Este se encontraram mais detalhados no relatório 3.

\section{Próximos passos}

Os próximos passos do projetos no que diz respeito ao software são:

\begin{itemize}
    \item Escrita de mais casos de teste
    \item Validar comportamentos de error como por exemplo queda da rede sem fio
    \item Melhorias na experiência do usuário no sistema
    \item Verificação junto ao cliente de como se calcula a quantidade de alimento baseado na taxa de conversão alimentar
    \item Desenvolvimento de cruzamento de dados. Por exemplo dados de crescimento versus dados de temperatura em um único gráfico.
    \item Teste de integração com o módulo de hardware completo
\end{itemize}
